% Template for ICASSP-2018 paper; to be used with:
%          spconf.sty  - ICASSP/ICIP LaTeX style file, and
%          IEEEbib.bst - IEEE bibliography style file.
% --------------------------------------------------------------------------
\documentclass{article}
\usepackage{spconf,amsmath,graphicx}
\graphicspath{{images/}}

% Example definitions.
% --------------------
% \def\x{{\mathbf x}}
% \def\L{{\cal L}}

\title{Neural Speech-To-Speech Synthesis}
\name{Jared Samet - UNI: jss2272}
\address{jss2272@columbia.edu}
\begin{document}
%\ninept
%
\maketitle
%
\begin{abstract}
  Prosody features that transfer across speakers can be extracted using unsupervised learning. By adding the cluster labels as annotation to text, a text-to-speech synthesis system can learn to incorporate the prosody into its output. The resulting system can produce audio that is different even when the text is the same and the resulting audio can mimic the prosody from the original speaker. The abstract should be about 175 words total.
\end{abstract}
%
\begin{keywords}
Prosody, unsupervised learning, speech synthesis, seq2seq
\end{keywords}
%
\section{Introduction}
\label{sec:intro}

I used Kaldi \cite{Povey_ASRU2011} to create an alignment for the Tedlium data using the final triphone model it created. For each vowel phoneme, I used Kaldi's pitch extractor and the first MFCC component (energy) to create two series of numbers. Kaldi's pitch extractor is already normalized but I used [StandardScaler] to normalize the energy component across the utterance. I fit a second-degree (?) Legendre polynomial to the pitch and power to create six features for each vowel. The duration gave me the seventh feature. I then ran K-means clustering on these to create eight different vowel clusters that were common across the entire range of speakers in the (subsampled) Tedlium data.

I then used the same triphone model to generate an alignment for the LJSpeech dataset, extracted the same pitch, power, and duration features for LJ, and used the previously computed vowel clusters to assign a cluster label to each vowel in the LJSpeech dataset. I (slightly) modified the Tacotron implementation to accept a sequence of tokens instead of a sequence of characters. Instead of text characters, my input tokens consisted of (Kaldi's) phonemes and the vowel cluster labels. Having suitably modified Tacotron, I then trained Tacotron on the [phoneme + cluster label, audio] pairs.

Finally, to see if it worked, I recorded myself saying a sentence in multiple ways, ran each .wav through the same align + label steps, and fed the resulting [phoneme+label, text] pairs to Tacotron. She said the same thing different ways.

\section{Related Work}
\label{sec:sota}

This project involved two main components: first, extracting prosody features from a set of input audio files; and second, training a text-to-speech synthesis model on a dataset that had been labeled using the extracted prosody features.

Selkirk \cite{selkirk1995sentence} discusses sentence prosody and pitch accent in the context of English. Although English is generally not thought of as a tonal language, Selkirk writes that ``[i]n English a pitch accent associates to a stress-prominent syllable in a word (typically the main word stress.)''
Ghahremani et al. \cite{ghahremani2014pitch} describes an pitch-extraction algorithm (``the Kaldi pitch tracker'') based on Talkin \cite{talkin1995robust} that is specifically designed for use in the speech recognition concept and is implemented in the open-source Kaldi project \cite{}. This project uses that implementation to extract the pitch contour.
Fujisaki \cite{fujisaki2004information} models the $F_0$ contour over the duration of an utterance as the sum of a set of impulse response and step response functions, parameterized with a finite number of scalar values.
Wang et al. \cite{wang2008mandarin} use the pitch and amplitude contours to improve tone recognition in Mandarin by identifying ``maxima, minima, and inflection points of particular acoustic events.''
Wong and Siu \cite{pui2004decision} use robust regression and orthogonal polynomials to create features for a decision tree classifier in order to recognize tones in Chinese lan guages. Finally, Lin \cite{lin2005language} and Mary \cite{mary2011extraction} use a small number of Legendre polynomial coefficients to represent the pitch contour as a finite-dimensional feature vector, which is the approach used in this project.

Speech synthesis or text-to-speech is a well-studied problem that has been actively researched since the 1950s. While there has been remarkable progress in the field in recent years, the quality of computer-generated speech has not yet reached human levels. Current commercial systems described in Khan et al. \cite{khan2016concatenative} and Taylor  \cite{taylor2009text} generally use concatenative speech synthesis to produce their output. However, the alternative approach of parametric synthesis using neural networks is rapidly gaining popularity, with several papers since 2016 demonstrating impressive results in the quality of the output.
The first of this generation was Google's WaveNet (Oord et al. \cite{oord2016wavenet}), followed in quick succession by Deep Voice and Deep Voice 2 from Baidu (Arik et al. \cite{arik2017deep}, \cite{arik2017deep2}), Char2Wav from MILA (Sotelo et al. \cite{sotelo2017char2wav}), and Tacotron from Google (Wang et al. \cite{wang2017tacotron}).
Each of these systems has taken a different approach to the network architecture to address different aspects of the speech synthesis pipeline. Tacotron, which is the backend used in this project, is a nearly end-to-end text-to-speech system based on the sequence-to-sequence with attention model. As the authors describe, ``The model takes characters as input and outputs the corresponding raw spectrogram, which is then fed to the Griffin-Lim reconstruction algorithm to synthesize speech.''

\section{Overview}
\label{sec:overview}
The goal of this project was to create a system that, given an input audio file from an arbitrary speaker, produces a synthesized audio output of the same utterance in the voice of a second speaker, where the prosody of the output audio matches that of the input audio as closely as possible. The system implemented uses a pipeline of several processing steps in order to accomplish this. An overview of the pipeline and a diagram are presented here for context; a detailed description of each step follows.

The first portion of the system is the vowel-cluster training process, which takes as input a previously-trained acoustic model for alignment and a speech dataset from multiple speakers. As output, it produces a clustering model that can be used to annotate an audio utterance from an arbitrary speaker with cluster labels for each vowel in the utterance. This portion of the system uses Kaldi to, first, perform forced alignment on the multi-speaker dataset, and, second, to extract the pitch contour and the first (energy) MFCC component for each frame of the input audio. Given the alignment, pitch, and power contours, an unsupervised clustering algorithm (K-means) trained on the audio segments corresponding to vowels to learn several distinct ways in which syllables can be pronounced.

The second portion of the system is the single-speaker annotation process, which takes as input the pre-existing acoustic model, the newly-trained vowel-cluster model, and a large speech dataset from a single speaker. As output, it produces a trained Tacotron model that can be used to generate synthesized utterances. This portion of the system first uses Kaldi to perform forced alignment on the speech dataset and extract the pitch and power features, as before. It then uses the vowel-cluster model to produce an annotated phoneme sequence for each utterance in the single-speaker dataset. Finally, the audio and the annotated phoneme sequence pairs are used to train the Tacotron model.

The final portion of the system is generates new utterances. As input, it takes the pre-existing acoustic model, the newly-trained vowel-cluster model, and the newly-trained Tacotron model, and an input audio file in the voice of an arbitrary speaker. As output, it produces synthesized audio of the equivalent utterance where the prosody matches that of the input utterance as closely as possible. This portion of the system computes an alignment for the input utterance and extracts the pitch and power features; uses the vowel cluster model to produce an annotated phoneme sequence for the input utterance; and, finally, uses the newly-trained Tacotron model to synthesize the output audio.

\begin{figure}[htb]

\begin{minipage}[b]{1.0\linewidth}
  \centering
  \centerline{\includegraphics[width=8.5cm]{Overall_Pipeline}}
\end{minipage}
\caption{Pipeline Overview}
\label{fig:overview}
\end{figure}

\section{Prosody Feature Extraction}
\label{sec:prosody}

All three portions of the pipeline involve extracting prosody features from the input audio -- in the vowel-cluster training process, the input audio is the multi-speaker dataset; in the single-speaker annotation process, the input audio is the single-speaker dataset; and in the utterance-generation portion, the input audio is the new utterance the user wishes to re-synthesize in a new voice.
The prosody feature extraction is performed in three steps.
First, Kaldi's \texttt{align.sh} script uses a pretrained acoustic triphone model -- in this project, the final triphone model resulting from Kaldi's TEDLIUM recipe -- to compute a forced alignment of the input audio, and Kaldi's \texttt{ali-to-phones} tool is used to convert the model-level alignment to a sequence of \textit{(phone\_id, start\_time, end\_time)} tuples.
Next, Kaldi's \texttt{make_mfcc_pitch.sh} script creates the MFCC and pitch features for each frame of the input audio and the \texttt{}


I used pitch (from Kaldi) and power. I used first three Legendre coefficients to extract a finite set of features for a phoneme of arbitrary length by using [-1, 1] as the domain regardless of the actual length of the phoneme. I added a few frames at the beginning and end in case Kaldi got the alignment wrong. Show what the Legendre coefficients look like for some different curves. Show what the actual pitch and power curves look like for some brief, manually labeled utterances. Describe the Kaldi pitch extractor. Definitely include the Kaldi paper as a reference. Explain how the Legendre coefficients work. Explain why these were a sensible way of capturing prosody. Talk about pitch envelopes and tonal languages.

\section{Clustering Vowels}
\label{sec:vowels}

Unsupervised learning FTW. Discuss why we should expect there to be clusters even in an atonal language like English. Talk about stressed vs unstressed as initial motivation but how there are probably more things like this. Talk about how stress is mostly a pitch change.

I ran K-means clustering on my seven features and created eight vowel clusters. I'm pretty sure I ran StandardScaler on the features first so K-means didn't get confused, double check this. This was intended to be enough to capture the major variation across how different vowels can be pronounced but not so many as to result in too-few training examples. I originally did this for each vowel separately, and only for a single speaker, but then I decided that was stupid so I did it across all speakers and across eight vowels. So there are only eight clusters. Here I need to demonstrate that the clusters are in fact "semantically" different in some way. Maybe include some metric of these or run TSNE on the coefficients.

I could probably have also just used the actual Legendre coefficients themselves but this would have required tinkering with the Tacotron internals more to accept continuous-valued features as part of the sequence instead of just a one-hot encoded value. This is something that could go in a future work section.

\section{Tacotron}
\label{sec:tacotron}

Describe the Tacotron architecture and explain why it was easy to add the cluster labels.

\subsection{Subheadings}
\label{ssec:subhead}

Just for reminder.

\subsubsection{Sub-subheadings}
\label{sssec:subsubhead}

Just for reminder.


\section{Results}
\label{sec:results}

Find some way to quantify that it actually did something beyond "she never stole my money".

Try and quantify that the different clusters are actually different in some way. This is probably the most important section. Quantify if they are different from male to female speakers in any way.

Try and quantify that the speech result is better for my Tacotron than for without annotations. Say why this could be useful even if no one wants to do speech to speech.

Try and quantify that the output is actually preserving stuff from the original speech dataset.

\section{Discussion}

\section{Limitations}

\section{Future Work}

\section{Acknowledgements}
\label{sec:acknowledgements}
I would like to thank Keith Ito for his outstanding open-source implementation of Tacotron. This project would not have been possible without his work. I would also like to thank Dan Povey, the lead developer of Kaldi, which was also essential to this project. Finally, I would like to thank Professor Beigi for teaching this class, which has been a pleasure!

\begin{figure}[htb]

\begin{minipage}[b]{1.0\linewidth}
  \centering
  \centerline{\includegraphics[width=8.5cm]{image1}}
%  \vspace{2.0cm}
  \centerline{(a) Result 1}\medskip
\end{minipage}
%
\begin{minipage}[b]{.48\linewidth}
  \centering
  \centerline{\includegraphics[width=4.0cm]{image3}}
%  \vspace{1.5cm}
  \centerline{(b) Results 3}\medskip
\end{minipage}
\hfill
\begin{minipage}[b]{0.48\linewidth}
  \centering
  \centerline{\includegraphics[width=4.0cm]{image4}}
%  \vspace{1.5cm}
  \centerline{(c) Result 4}\medskip
\end{minipage}
%
\caption{Example of placing a figure with experimental results.}
\label{fig:res}
%
\end{figure}


% To start a new column (but not a new page) and help balance the last-page
% column length use \vfill\pagebreak.
% -------------------------------------------------------------------------
%\vfill
%\pagebreak


\vfill\pagebreak

\bibliographystyle{IEEEbib}
\bibliography{strings,refs}

\end{document}
